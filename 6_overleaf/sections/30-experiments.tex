\section{Methods}
\label{sec:method}

% \subsection{Baseline Methods}
To compare the results of our novel method, we used a number of baseline values, including both CoT and Direct Prompting. We use GPT-4o \citep{OpenAI2024-em} to show SoTA zero-shot Direct Prompting results, and mistral-small \citep{mistral_small_24b_2025} for Direct Prompting, CoT, and our novel method.

For each method, answers are generated for 4 permutations of each of the 1,700 questions. The questions are answered with the correct answer being located in A, B, C, and D positions. By contrasting the performance between permutations we generate the results shown in Section~\ref{sec:results}.

\subsection{Cognitive Alignment for MCQ Answering}

We present Cognitive Alignment(CA) as a method of fully removing order bias from the process of LLM MCQ answering. The process begins by using the LLM to answer the given question without any access to the provided choices. Next, both the LLM answer and the 4 possible answers are embedded using an embedding model. \footnote{In our experiment we use the distiluse-base-multilingual-cased-v2 (cite) embedding model because of its training in both English and Chinese.} Finally, the cosine similarity between the embedded LLM answer and the 4 answer choices is calculated, and the answer choice with the highest similarity is selected as the predicted answer.

\begin{equation}
\text{Answer} = \max_{i \in \{1, 2, 3, 4\}} \text{sim}(\text{LLM}, A_i)
\end{equation}


By never providing the answer choices to the LLM directly, order is fundamentally never considered. This inherently removes the ability for a given order to influence the LLM.

\paragraph{CA-Adjusted MCQ Answering}

While the pure CA approach completely removes the possibility of order bias, it has a major drawback- a large decrease in accuracy. Our results show pure CA on our English and Chinese datasets have an accuracy of 43\% and 38.333\% respectively, a drop from Direct Prompting of 38.044\% and 35\% (respectively). With such a major loss, we propose a second novel method, CA-Adjusted MCQ Answering, to balance the benefits of Direct Prompting accuracy and the reduction in order bias of CA.

CA-Adjusted MCQ Answering works by removing potential answer choices as selections based on what CA judges as "worse" answers. Given a set of similarity scores for A, B, C, D answer choices, any choice within a certain distance (cutoff value) from the highest similarity value are replaced with "DO NOT PICK THIS OPTION" in the dataset, reducing the distractions provided to the LLM when answering.

For example, consider the following:

\noindent\textbf{Question} \\
The soils of which of the following biomes has the highest rate of leaching and cycling of nutrients?

\vspace{0.5em}

\noindent\textbf{Answer choices}
\begin{enumerate}[label=\Alph*., itemsep=0pt, topsep=0pt]
    \item Tropical rain forest
    \item Tundra
    \item Taiga
    \item Desert
\end{enumerate}


\vspace{0.5em}

\noindent\textbf{Free-response answer generated by CA method:} \\
\textit{Tropical Rainforest}

\vspace{0.5em}

\noindent\textbf{Similarity scores generated by CA method}

\begin{center}
\begin{tabular}{@{}ll@{}}
\toprule
\textbf{Choice} & \textbf{Similarity Score} \\
\midrule
A & 0.96880 \\
B & 0.17935 \\
C & 0.19457 \\
D & 0.34539 \\
\bottomrule
\end{tabular}
\end{center}

\vspace{0.5em}

\noindent\textbf{New potential answer choices provided with a cutoff value of 0.459}
\begin{enumerate}[label=\Alph*., itemsep=0pt, topsep=0pt]
    \item Tropical rain forest
    \item DO NOT PICK THIS OPTION
    \item DO NOT PICK THIS OPTION
    \item DO NOT PICK THIS OPTION
\end{enumerate}


By removing low-confidence answers, the LLM is less likely to choose incorrect answers due to order bias, and by adjusting the cutoff value, we can balance the lower order bias of CA and the better accuracy of Direct Prompting.

We choose the cutoff value 0.459 in our testing as in every case where the difference between the highest and second highest similarity value is greater than or equal to 0.459 there is a 100\% accuracy (the correct answer is always chosen in the pure CA strategy in these cases).

% \subsection{Iterative Elimination for MCQ Answering}
% The first idea!
% do this if we have to do it like this!