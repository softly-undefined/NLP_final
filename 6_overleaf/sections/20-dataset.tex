\section{Test Set}
\label{sec:dataset}

For this experiment, we use both English and Chinese versions of the MMLU data set introduced in \citet{hendryckstest2021}. For Chinese, we use a translated version of the benchmark made available by OpenAI at \url{https://openaipublic.blob.core.windows.net/simple-evals/mmlu_ZH-CN.csv}. Throughout the cleaning process we preserve row alignment between the two languages to ensure identical questions are being considered.

\subsection{Data Cleaning}


A total of 176 rows of our bilingual dataset were found to have different correct answer values for the two languages, which were removed to reduce complexity.

We next divide the data set into subcategories following the structure in \citet{hendryckstest2021}. The subcategories are as follows: biology, business, chemistry, computer science, culture, economics, engineering, geography, health, history, law, math, other, philosophy, physics, politics, and psychology. From each subcategory, we sample 100 aligned English-Chinese question pairs, resulting in a total of 1,700 questions.